\chapter{Introduction}

\section{Background and Recent Research}
\subsection{Background}
Psychobiology researchers in Bogazici University conduct extensive animal tracking experiments. These experiments are sometimes conducted with experiment automation tools like EthoVision from Noldus Software. But such software currently in use has the limitation of daylight lighting conditions.

To be able to conduct experiments that span over the full day/night cycle, they were in need of systems capable of object tracking under no light conditions.

\subsection{Literature Survey}

\subsubsection{EthoVision Video Tracking System \cite{Spink2001731}}
Video tracking systems enable behavior to be studied in a reliable and consistent way, and over longer time periods than if they are manually recorded. The system takes an analog video signal, digitizes each frame, and analyses the resultant pixels to determine the location of the tracked animals (as well as other data). Calculations are performed on a series of frames to derive a set of quantitative descriptors of the animal's movement. 
\subsubsection{Validation of a digital video tracking system for recording pig locomotor behaviour \cite{Lind2005123}}
The authors are introducing a system for automatically tracking pig locomotor behaviour. Transposing methods for the video-based tracking of rodent behaviour engenders several problems. The authors propose a method which improves existing methods, based on image-subtraction, to offer increased flexibility and accuracy in tracking large-sized animals in situations with a constantly changing background.
\subsubsection{Automatic real-time monitoring of locomotion and posture behaviour of pregnant cows prior to calving using online image analysis \cite{Cangar200853}}
Monitoring the locomotion and posture behaviour of pregnant cows close to calving is essential in determining if there is a need for human intervention to assist parturition. In this study an automatic real-time monitoring technique is described in detail which allows identifying the locomotion and posturing behaviour of pregnant cows prior to calving.

\section{Motivation}

This project aims to provide researchers with a tool to create, repeat and document animal tracking experiment results in a credible way.

It is meant to provide users with unbiased and structured behaviour analysis reports.
